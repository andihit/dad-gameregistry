\chapter{Datos generales}

\section{Miembros del grupo}

\begin{table}[htdp]
\begin{center}
\begin{tabular}{|l|l|l|c|}
\hline
\textbf{Apellidos}&\textbf{Nombre}&\textbf{Correo-e}&\textbf{Grupo}\\
\hline
Gálvez-Cañero&Rafael&\href{mailto:galvesband@gmail.com}{galvesband@gmail.com}&18\\
Gerstmayr&Andreas&\href{mailto:andreas.gerstmayr@gmail.com}{andreas.gerstmayr@gmail.com}&18\\
\hline
\end{tabular}
\end{center}
\caption{Miembros del grupo}
\label{tab:miembros}
\end{table}%


\section{Descripción del sistema}

\begin{itemize}
\item \textbf{Tipo de sistema distribuido}:
\item \textbf{Nombre del proyecto}: Plataforma de juegos, Game Register
\item \textbf{Breve descripción}: Sub-sistema para registrar sesiones de juego e información asociada.
\end{itemize}

\subsection{Funcionalidad observable}

\begin{itemize}
\item Registrar el inicio y el término de todas las sesiones de juego.
\item Visualizar el historial de juegos.
\item Visualizar qué jugadores juegan en este momento.
\end{itemize}

\subsection{Servicios ofrecidos}
\begin{itemize}
\item Servicio de Registro: Capacidad de aceptar la información de una sesión de juego.
\item Servicio de Historial: Ofrece métodos para consultar el historial de sesiones.
\item Servicio de Sesiones Online: Muestra una lista de jugadores actualmente en activo.
\end{itemize}

\subsection{Servicios demandados}
\begin{itemize}
\item Servicio X: breve descripción. Fecha aproximada a partir de la que se necesitará (si se conoce).
\item Servicio Y: idem.
\end{itemize}

\section{Direcciones de descarga y planificación}

\begin{table}[htdp]
\begin{center}
\begin{tabular}{|c|c|}
\hline
\textbf{Código fuente}&\url{https://repositorio.informatica.us.es/svn/lq3vqrtzfnh2nx9yhpk}\\
\hline
\multicolumn{2}{|c|}{\textbf{Planificación temporal}}\\
\hline
Iteración 1&17/02/2015\\
Iteración 2&01/03/2015\\
Iteración 3&15/03/2015\\
Iteración 4&05/04/2015\\
Iteración 5&19/04/2015\\
Iteración 6&10/05/2015\\
Iteración 7&24/05/2015\\
Entrega Final&07/06/2015\\
\hline
\end{tabular}
\end{center}
\caption{Datos generales del trabajo en grupo}
\label{tab:datosgenerales}
\end{table}%

\section{Seguimiento}

\begin{table}[htdp]
\begin{center}
\begin{tabular}{|c|c|c|c|c|c|c|c|c|c|c|c|}
\cline{2-10}
\multicolumn{1}{c}{}&\multicolumn{9}{|c|}{\textbf{Iteración}}&\multicolumn{2}{c}{}\\
\hline
\textbf{Estudiante}&1&2&3&4&5&6&7&8&Final&Total&Pond.\\
\hline
Rafael Gálvez-Cañero&5&5&5&5&5&5&5&5&5&\textbf{40}&1\\
Andreas Gerstmayr&5&5&5&5&5&5&5&5&5&\textbf{40}&1\\
\hline
Total&30&30&30&30&30&30&30&30&\multicolumn{2}{c}{}\\
\cline{1-9}
\end{tabular}
\end{center}
\caption{Tabla de seguimiento}
\label{tab:seguimiento}
\end{table}%
