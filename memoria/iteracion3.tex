\chapter{Iteración 3}
\section{Objetivos de iteración}
\begin{itemize}
  \item Integración de Gradle 
  \item Servidor dockerizado
  \item Estructura inicial del Cliente vertx.
\end{itemize}

\section{Gradle}
\emph{Gradle} es un gestor de construcción especialmente indicado para proyectos
Java y con soporte para \emph{Groovy}, \emph{Vertx} y \emph{Maven}.

Ha sido integrado mediante el wrapper \texttt{gradlew} que permite utilizar Gradle sin
instalarlo de forma global en el sistema de desarrollo. La primera vez que se
lanza descargará todas las bibliotecas necesarias.

En el archivo \textbf{Readme.md} hay información básica sobre como construir el
proyecto. Algunos comandos útiles:

\begin{itemize}
 \item Para \textbf{construir} el proyecto: \\
       \texttt{\$ ./gradlew clean modZip}
 \item Para \textbf{lanzar} el servidor en la máquina local: \\
       \texttt{\$ ./gradlew runMod -i}
 \item Para lanzar los \textbf{tests}: \\
       \texttt{\$ ./gradlew clean test}
 \item Para preparar el proyecto para un \textbf{entorno de desarrollo}:
    \begin{itemize}
      \item Eclipse: \texttt{\$ ./gradlew eclipse}
      \item IDEA: \texttt{\$ ./gradlew idea}
    \end{itemize}
\end{itemize}

\section{Dockerización}
\emph{Docker} es una tecnología que permite utilizar contenedores sobre \emph{Linux} para
ejecutar procesos de forma aislada y con un runtime reproducible.

Nuestro proyecto proporciona un archivo \textbf{Dockerfile} con las instrucciones necesarias
para construir el contenedor de la aplicación. Además, el archivo \textbf{Readme.md} contiene
información sobre el procedimiento para lanzar el proyecto en \emph{Docker}. 

El procedimiento para lanzar el servidor con \emph{Docker} ahora mismo es el siguiente:

\begin{itemize}
 \item Limpiar y contruir el proyecto: \\
       \texttt{\$ ./gradlew clean modZip}
 \item Construir el contenedor con el servidor: \\
       \texttt{\$ docker build -t distributedsystems/gameregistry .} \\
       (notar el punto final del comando que indica a \emph{Docker} dónde buscar el archivo
       \emph{Dockerfile} con las instrucciones de construcción). La construcción incluye el 
       resultado de compilación del proyecto por lo que cada vez que este cambie el contenedor
       debe ser reconstruido.
 \item Lanzar el contenedor del servidor: \\
       \texttt{\$ docker run distributedsystems/gameregistry}
\end{itemize}
